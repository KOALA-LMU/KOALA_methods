\documentclass[10pt,xcolor=dvipsnames,t,headinclude,headsepline=1.5cm,usepdftitle=false]{beamer}

\usepackage{framed}
\usepackage{setspace}
\usepackage{color}
\usepackage{amsmath}
\usepackage{booktabs}
\usepackage{bm}
% \usepackage{Sweave}
\usepackage{graphicx}
\usepackage{centernot}
% \usepackage{commath}
\usepackage{epstopdf}
\usepackage{caption}
\usepackage{pifont} % Symbols
\usepackage{makecell} % for line breaks in tables
\usepackage{array}
\usepackage{hyperref}
\usepackage{appendixnumberbeamer} % for \appendix, starting page counting anew from 1 for the appendix
\usepackage{multirow}
\usepackage{tabularx}

% New tabular column types
\newcolumntype{Z}{>{\centering\arraybackslash}X} % centered tabularx columns
\newcolumntype{P}[1]{>{\centering\arraybackslash}p{#1}} % like p, just centered
\newcolumntype{M}[1]{>{\centering\arraybackslash}m{#1}} % M vertically centers the content


% Theme
\usetheme{Boadilla}
\usecolortheme[named=MidnightBlue]{structure}
\definecolor{MidnightBlue}{rgb}{0.22,0.38,0.58} 
\usefonttheme{default}

\setbeamercolor{shadebox}{bg=MidnightBlue!20}

% Table of contents
\setcounter{tocdepth}{2}
% Prettify the section number in the table of contents
\setbeamertemplate{section in toc}{\inserttocsectionnumber.~\inserttocsection}

\setbeamercovered{transparent=25}
\setbeamercovered{still covered={\opaqueness<1->{0}},again covered={\opaqueness<1->{10}}}
\setbeamercovered{dynamic}
%\setbeameroption{show notes on second screen=left}
%\setbeameroption{hide only notes}
%\setbeamersize{text margin left=1cm,text margin right=10pt}
\setbeamercovered{transparent}
\setbeamertemplate{frametitle continuation}{\gdef\beamer@frametitle{}}
\setbeamertemplate{footline}{\vspace{-1cm}{\line(1,0){345}}}

    \makeatletter
    \renewenvironment{thebibliography}[1]
         {%\section*{\refname}% <--- outcommented
          \@mkboth{\MakeUppercase\refname}{\MakeUppercase\refname}%
          \list{\@biblabel{\@arabic\c@enumiv}}%
               {\settowidth\labelwidth{\@biblabel{#1}}%
                \leftmargin\labelwidth
                \advance\leftmargin\labelsep
                \@openbib@code
                \usecounter{enumiv}%
                \let\p@enumiv\@empty
                \renewcommand\theenumiv{\@arabic\c@enumiv}}%
          \sloppy
          \clubpenalty4000
          \@clubpenalty \clubpenalty
          \widowpenalty4000%
          \sfcode`\.\@m}
         {\def\@noitemerr
           {\@latex@warning{Empty `thebibliography' environment}}%
          \endlist}
    \makeatother

    \makeatletter
\newbox\@backgroundblock
\newenvironment{backgroundblock}[2]{%
  \global\setbox\@backgroundblock=\vbox\bgroup%
    \unvbox\@backgroundblock%
    \vbox to0pt\bgroup\vskip#2\hbox to0pt\bgroup\hskip#1\relax%
}{\egroup\egroup\egroup}
\addtobeamertemplate{background}{\box\@backgroundblock}{}
\makeatother

% Literatureinbindung
\usepackage[backend=bibtex, %alternativ: biber
            style=authoryear,
            citestyle=authoryear,
            dashed=false] % bei mehreren Quellen von einem Autor soll immer der Autorenname wiederholt werden im Literaturverzeichnis, und nicht ab der 2.Quelle ein Dash "-" gemacht werden statt dem Autornamen
{biblatex}
\addbibresource{../bauer_references.bib}
\DefineBibliographyStrings{ngerman}{
  andothers = {{et\,al\adddot}}, % bei mehreren Autoren 'et al.' anzeigen, ansonsten wäre es 'u.a.'
}

% Show code in slides
\usepackage{listings}
\lstset{ 
  language=R,                     % the language of the code
  basicstyle=\scriptsize\ttfamily, % the size of the fonts that are used for the code
  numbers=none,                   % where to put the line-numbers (z.B. 'numbers=left')
  numberstyle=\footnotesize\color{Blue},  % the style that is used for the line-numbers
  stepnumber=1,                   % the step between two line-numbers. If it's 1, each line will be numbered
  numbersep=5pt,                  % how far the line-numbers are from the code
  backgroundcolor=\color{gray!3},  % choose the background color. You must add \usepackage{color}
  showspaces=false,               % show spaces adding particular underscores
  showstringspaces=false,         % underline spaces within strings
  showtabs=false,                 % show tabs within strings adding particular underscores
  frame=single,                   % adds a frame around the code (z.B. 'frame=single')
  rulecolor=\color{black},        % if not set, the frame-color may be changed on line-breaks within not-black text (e.g. commens (green here))
  xleftmargin=0cm,
  tabsize=2,                      % sets default tabsize to 2 spaces
  captionpos=b,                   % sets the caption-position to bottom
  breaklines=true,                % sets automatic line breaking
  breakatwhitespace=false,        % sets if automatic breaks should only happen at whitespace
  keywordstyle=\color{Black},      % keyword style (z.B. 'RoyalBlue')
  commentstyle=\color{Green},   % comment style
  stringstyle=\color{Green}      % string literal style
}

% gifs
\epstopdfDeclareGraphicsRule{.gif}{png}{.png}{convert gif:#1 png:\OutputFile}
\AppendGraphicsExtensions{.gif}

\colorlet{shadecolor}{MidnightBlue!20}

% No Figure in caption
\setbeamertemplate{caption}{\scriptsize\insertcaption\normalfont}

% Further settings
\beamertemplatenavigationsymbolsempty
\setbeamertemplate{itemize item}{$\bullet$}
\setbeamertemplate{itemize subit3em}{--}
\setbeamertemplate{enumerate items}[default]
\setlength{\abovedisplayskip}{0pt}
\setlength{\belowdisplayskip}{0pt}
\setlength{\abovedisplayshortskip}{0pt}
\setlength{\belowdisplayshortskip}{0pt}

% Header customization
\setbeamerfont{frametitle}{size = \large, series = \bfseries}

% ----- Newcommands
% Colors
\newcommand{\red}[1]{\textcolor{red}{#1}}
\newcommand{\redbf}[1]{\textbf{\textcolor{red}{#1}}}
\newcommand{\blue}[1]{\textcolor{MidnightBlue}{#1}}
\newcommand{\bluebf}[1]{\textbf{\textcolor{MidnightBlue}{#1}}}
\newcommand{\darkgray}[1]{\textcolor{darkgray}{#1}}
\newcommand{\darkgraybf}[1]{\textbf{\textcolor{darkgray}{#1}}}
\newcommand{\shade}[1]{\textcolor{MidnightBlue!70}{#1}}
\newcommand{\shadesmall}[1]{{\footnotesize \textcolor{MidnightBlue!70}{#1}}}
\newcommand{\gray}[1]{\textcolor{gray!70}{#1}}
\newcommand{\graysmall}[1]{{\footnotesize \textcolor{gray!70}{#1}}}
% LMU colors
\definecolor{LMUgreen}{RGB}{0, 148, 64}		% #fce94f
\newcommand{\LMU}[1]{\textcolor{LMUgreen}{#1}}
\newcommand{\LMUbf}[1]{\textbf{\textcolor{LMUgreen}{#1}}}
\definecolor{LMUdarkgray}{RGB}{127, 127, 127}		% #fce94f
\newcommand{\LMUdarkgray}[1]{\textcolor{LMUdarkgray}{#1}}
\newcommand{\LMUdarkgraybf}[1]{\textbf{\textcolor{LMUdarkgray}{#1}}}
\definecolor{LMUlightgray}{RGB}{192, 192, 192}		% #fce94f
\newcommand{\LMUlightgray}[1]{\textcolor{LMUlightgray}{#1}}

% Individual frametitle
\newcommand{\ft}[1]{\frametitle{#1 \\ \vspace{-0.3cm}\hspace{0cm}{\rule{\textwidth}{0.5mm}}}}


% ----- Further settings
% Special packages
\usepackage{animate} % for including .gifs
\usepackage{colortbl} % for coloring \hline's in tabulars
% Footer
\setbeamertemplate{footline}{\colorbox{MidnightBlue}{\textcolor{white}{\quad Alexander Bauer \hspace{10.3cm}\insertframenumber \ / \inserttotalframenumber \hspace{2cm}}}
\beamertemplatenavigationsymbolsempty
}
% Titlepage
\title[Bauer_DAGStat]{\ \\[0.5cm] {\large \bf KOALA: A new paradigm for election coverage} \\[0.2cm] {\normalsize An opinion poll based ``now-cast'' of probabilities of events in \\ multi-party electoral systems \\[0.25cm]}}
\subtitle[Kurzform]{
\vspace{-0.3cm}
\rule{0.85\textwidth}{0.3mm} \\[0.45cm]
{\normalsize Alexander Bauer} \\
{\scriptsize Statistical Consulting Unit StaBLab, LMU Munich} \\[0.25cm]
{\footnotesize DAGStat \textbar \ March 20, 2019 \textbar \ Munich}
}
\author[A. Bauer]{\vspace{-1cm}
\normalfont}
\date{}
% ----- End of settings


\begin{document}

\frame{
\vspace{1cm}
\titlepage
}

% Second titlepage
\title[Bauer_DAGStat]{\\[11px]
\includegraphics[width=0.35\textwidth]{figures/koala} \\[0.1cm]}
\subtitle[Kurzform]{
\vspace{-0.3cm}
\rule{0.85\textwidth}{0.3mm} \\[0.45cm]
{\normalsize Alexander Bauer} \\
{\scriptsize Statistical Consulting Unit StaBLab, LMU Munich} \\[0.25cm]
{\footnotesize DAGStat \textbar \ March 20, 2019 \textbar \ Munich}
}
\author[A. Bauer]{\vspace{-1cm}
\normalfont}
\date{}

\addtocounter{framenumber}{-1}
\frame{
\vspace{1cm}
\titlepage
}

% Third titlepage
\title[Bauer_DAGStat]{\\[11px]
\includegraphics[width=0.35\textwidth]{figures/koala} \\[0.1cm]}
\subtitle[Kurzform]{
\vspace{-0.3cm}
\rule{0.85\textwidth}{0.3mm} \\[0.45cm]
}
\author[A. Bauer]{\vspace{-1cm}
\normalfont}
\date{}

\addtocounter{framenumber}{-1}
\frame{
\vspace{-11px}
\titlepage
\vspace{-95px}
\hspace{17px}
\begin{tabular}{ll}
\bluebf{\footnotesize Collaborators} & \\
\blue{\footnotesize Dr. Andreas Bender} & \blue{\footnotesize Nuffield Department of Clinical Medicine,} \\
& \blue{\footnotesize University of Oxford, United Kingdom} \\
\blue{\footnotesize Dr. Andr\'e Klima} & \blue{\footnotesize StaBLab, LMU Munich} \\
\blue{\footnotesize Prof. Dr. Helmut K\"uchenhoff} & \blue{\footnotesize StaBLab, LMU Munich} \\
\end{tabular}
}

% ----- Outline
\frame{\ft{Outline}
  \tableofcontents[subsectionstyle=show/show/hide]
}
\addtocounter{framenumber}{-1}


% ----- Motivation
\section{Motivation}
\frame{\ft{Outline}
  \tableofcontents[currentsection, subsectionstyle=show/show/hide]
}

\frame{\ft{\large{\blue{\ding{202}}} Motivation}
\begin{center}
\includegraphics[width=0.5\textwidth]{figures/motivation_forsa_130920}
\end{center}
\bigskip
\textbf{Questions of interest}
\begin{itemize}
  \item Which parties will pass the $5\%$ hurdle and enter the parliament?
  \item Which parties will form the governing coalition?
  \item Which party will have the third largest share of votes?
\end{itemize}}

\addtocounter{framenumber}{-1}
\frame{\ft{\large{\blue{\ding{202}}} Motivation}
\textbf{Reported voter shares} \\[0.37cm]
\begin{center}
{\footnotesize
\begin{tabular}{cccccccc}
\toprule[0.09 em]
Union & SPD & Greens & FDP & The Left & Pirates & AfD & Others \\
\midrule
40\% & 26\% & 10\% & 5\% & 9\% & 2\% & 4\% & 5\% \\
\bottomrule[0.09 em]
\end{tabular}
}
\end{center}
\medskip
\textbf{Redistributed voter shares} {\footnotesize (based on $5\%$ hurdle)} \\[0.37cm]
\begin{center}
{\footnotesize
\begin{tabular}{cccccccc}
\toprule[0.09 em]
Union & SPD & Greens & FDP & The Left & Pirates & AfD & Others \\
\midrule
44.44\% & 28.89\% & 11.11\% & 5.56\% & 10.00\% & - & - & - \\
\bottomrule[0.09 em]
\end{tabular}
}
\end{center}
\bigskip
\pause
\begin{itemize}
  \item Union-FDP have a joint seat share of exactly $50\%$
  \item Stating that Union-FDP would thus miss a joint majority would neglect sample uncertainty \\ \ 
  \item[$\Rightarrow$] We calculate event probabilities that fully reflect sample uncertainty
\end{itemize}
}

\frame{\ft{\large{\blue{\ding{202}}} Motivation}
\vspace{30px}
\bluebf{We aim to do} \textbf{now-casting}
\begin{itemize}
  \item We incorporate the uncertainty as reported by the polling agencies
  \item Potential house biases or an industry bias are not accounted for
\end{itemize}
\bigskip \bigskip
\pause
\bluebf{We} \redbf{do not} \bluebf{aim to do} \redbf{for-casting}
\begin{itemize}
  \item Our approach simply communicates sample uncertainty in a novel way
  \item Also, a relevant share of voters is still undecided shortly before election day (Küchenhoff et al., 2018)
\end{itemize}
}

% ----- Methods
\section{Methods}
\frame{\ft{Outline}
\tableofcontents[currentsection, subsectionstyle=show/show/hide]
}

\frame{\ft{\large{\blue{\ding{203}}} Methods}
\textbf{Estimating probabilities of events (POEs)} \\[0.1cm]
Given one opinion poll with sample size $n$:
$$
\boldsymbol{X} = (X_1,\ldots, X_P)^T \sim Multinomial(n, \theta_1,\ldots, \theta_P),
$$
with voter counts $X_j$ and the true percentage of voters $\theta_j$ per party $j$ \\
{\footnotesize (assuming a simple random sample, ignoring a possible bias)}
\\[0.5cm]
\pause
Using an \bluebf{uninformative Dirichlet prior}
% for the true party shares
(Gelman et al., 2013)
$$
\begin{aligned}
\boldsymbol{\theta} &= (\theta_1,\ldots,\theta_P)^T \sim Dirichlet(\alpha_1,\ldots,\alpha_P), \\
\text{with} &\ \ \ \ \ \ \ \ \ \ \ \ \ \ \ \alpha_1 = \ldots = \alpha_P = \frac{1}{2},
\end{aligned}
$$
\ \\[0.1cm]
the resulting posterior distribution of $\boldsymbol{\theta} | \boldsymbol{x}$ is again Dirichlet:
$$
\boldsymbol{\theta}|\mathbf{x} \sim Dirichlet(x_1 + 1/2,\ldots, x_P + 1/2).
$$
}

\frame{\ft{\large{\blue{\ding{203}}} Methods}
\textbf{Estimating probabilities of events (POEs)} \\[0.5cm]
Given the \bluebf{posterior distribution of voter shares} we can use \\ 
\textbf{Monte Carlo simulations} to estimate POEs: \\[0.1cm]
\begin{enumerate}
  \item Simulate $10\,000$ election outcomes from the posterior
  \item If necessary: Redistribute voter shares to get obtained seats in parliament
  \item POE $=$ Percentage of simulations where event of interest occurred
\end{enumerate}
\bigskip \bigskip
\pause
\bluebf{Example} \\[0.1cm]
Given the Forsa poll, the coalition of Union-FDP obtained a majority of seats in
$2\,633$ of $10\,000$ simulations \\[0.2cm]
$\Rightarrow \text{POE} \approx 26\%$
}

\frame{\ft{\large{\blue{\ding{203}}} Methods}
\textbf{Voter shares}
\\[1.0cm]
\begin{center}
\includegraphics[width=0.5\textwidth]{figures/motivation_forsa_130920_bw}
\end{center}
}

\frame{\ft{\large{\blue{\ding{203}}} Methods}
\textbf{Posterior distribution} of joint CDU-FDP seat share
\\[1.5cm]
\begin{center}
\includegraphics[width=0.6\textwidth]{figures/forsa_posterior}
\end{center}
$\Rightarrow \text{POE} \approx 26\%$
}

\frame{\ft{\large{\blue{\ding{203}}} Methods}
\begin{center}
\includegraphics[width=0.7\textwidth]{figures/screenshot}
\end{center}
Plan: \\
1. Als Motivationsbeispiel die frueheste gepoolte Umfrage im 2013er Ridgeline Plot nehmen \\
2. Erstmal nur Ridgeline zeigen und Zeitverlauf animieren \\
3. Bei erster richtiger Bimodalitaet Animation anhalten und kurz daneben Union- und FDP-Stimmanteil-Zeitverlauf einblenden \\
4. Animation fertig laufen lassen (Union und FDP dabei wieder ausgeblendet) \\
5. Am Ende links den redistributed joint voter share und die POE-W'keiten einblenden
}

\frame{\ft{\large{\blue{\ding{203}}} Methods}
\textbf{Pooling} \\[0.1cm]
We aggregate multiple polls to reduce sample uncertainty. \\
In case of multiple random samples:
\\[0.2cm]
$$
\left( \sum\limits_i X_{i1},\ldots, \sum\limits_i X_{iP} \right)^T
 \sim Multinomial \left( \sum\limits_i n_i,\theta_1,\ldots,\theta_P\right).
$$
\ \\[0.3cm]
We account for correlations between polling agencies by using an \\
\bluebf{effective sample size} (Hanley et al., 2003).
\\[0.7cm]
\pause
$\Rightarrow$ \bluebf{Example:} Pooling two polls with $1\,500$ and $2\,000$ respondents \\
\textcolor{white}{$\Rightarrow$ \bf Example:} {\footnotesize (where the strongest party obtained $40\%$)}, \\
\textcolor{white}{$\Rightarrow$ \bf Example:} we get a conservative effective sample size of $n_{\text{eff}} = 2\,341$.
}

\frame{\ft{\large{\blue{\ding{203}}} Methods}
\textbf{Pooling in practice} \\[0.1cm]
\begin{itemize}
  \item We only pool surveys published in the last $14$ days
  \item We only include one survey per polling agency
\end{itemize}
\bigskip \bigskip
\pause
\textbf{Correction of rounding errors} \\[0.1cm]
Party shares are only published with a certain accuracy. \\
We add \bluebf{uniformly distributed random noise} to avoid potential biases.
}

% ----- Technical implementation
\section{Technical implementation}
\frame{\ft{Outline}
\tableofcontents[currentsection, subsectionstyle=show/show/hide]
}

\frame{\ft{\large{\blue{\ding{203}}} Technical implementation}
\textbf{R package \texttt{coalitions}} \\[0.1cm]
\includegraphics[width=0.1\textwidth]{figures/coalitions_hex}
...
Code example:
% \begin{lstlisting}
% 2+2
% \end{lstlisting}
More on GitHub
}

\frame{\ft{\large{\blue{\ding{203}}} Technical implementation}
\textbf{Web-Interface}
\begin{itemize}
  \item koala.stat.uni-muenchen.de
  \item Blog
  \item @koala\_lmu \includegraphics[height=3ex]{figures/twitter_dark}
  \item based on Shiny
  \item automatic update scraping data from \texttt{wahlrecht.de}
\end{itemize}
}



% ----- Conclusion
\section{Results \& Outlook}
\frame{\ft{Outline}
\tableofcontents[currentsection, subsectionstyle=show/show/hide]
}

\frame{\ft{\large{\blue{\ding{206}}} Results}
content
}


% Materials
\addtocounter{framenumber}{-1}
\frame{\ft{References}
\bluebf{Topic} \\[0.07cm]
{\footnotesize
\textbf{Doe J, Mustermann M (2019)} \textcolor{darkgray}{This is the paper title. Journal, 19(2--3), 1--19} \\[0.04cm]
\textbf{Doe J, Mustermann M (2019)} \textcolor{darkgray}{This is the paper title. Journal, 19(2--3), 1--19} \\[0.04cm]
} \ \\[0.75cm]
\bluebf{Another topic} \\[0.07cm]
{\footnotesize
\textbf{Doe J, Mustermann M (2019)} \textcolor{darkgray}{This is the paper title. Journal, 19(2--3), 1--19} \\[0.04cm]
Hanley et al., 2003
}
}

\addtocounter{framenumber}{-1}
\frame{\ft{References}
\bluebf{One more topic} \\[0.07cm]
{\footnotesize
\textbf{Doe J, Mustermann M (2019)} \textcolor{darkgray}{This is the paper title. Journal, 19(2--3), 1--19} \\[0.04cm]
}
}

\end{document}